\section{Bibliografia}
CHINNICI GIORGIO, “Guarda Caso. i meccanismi segreti della meccanica quantistica” - Hoepli (2017)\\
Il libro espone i concetti della meccanica quantistica in una via di mezzo tra il formalismo scientifico e la divulgazione. L’ho usato principalmente nel secondo capitolo della tesina: “Cenni di meccanica quantistica”\\
\section{Sitografia}
https://quantumexperience.ng.bluemix.net/qx/experience\\
Si tratta dell’interfaccia per l’utilizzo di un computer quantistico (situato negli Stati Uniti), e contiene guide all’utilizzo e al concetto generale di computazione quantistica\\
\\
www.youtube.com\\
Vari video di divulgazione a riguardo di computer e computazione quantistica\\
\\
https://github.com/QISKit/qiskit-sdk-py\\
Framework contenente un simulatore di computer quantistico e interprete del linguaggio macchina\\
\\
https://www.dwavesys.com/\\
Sito di una delle maggiori start-up a riguardo dei computer quantistici; presente come vengono costruiti\\
\\
https://www.scottaaronson.com/blog/?p=208\\
Spiegazione dell’algoritmo di Shor a cura di un Scott Aaronson, professore al MIT e  esperto in computazione quantistica