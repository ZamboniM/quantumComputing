\section{Cos'è la meccanica quantistica}
La meccanica quantistica è una teoria fisica che descrive il comportamento della materia, delle radiazioni e delle loro interazioni. In particolare studia i fenomeni microscopici e fa parte del \textit{modello standard} (cioè la teoria fisica che descrive tre delle quattro forze fondamentali: interazioni forte, debole, elettromagnetica e tutte le particelle collegate. Essa è quindi incompatibile solo alla relatività generale che spiega la forza di gravità)

\section{Elementi fondamentali della meccanica quantistica}
\subsection{Dualismo onda-particella}
Nella fisica classica vigevano due blocchi di leggi distinti e apparentemente indipendenti: quelle di Newton, che descrivono i corpi meccanici, e quelle di Maxwell, che invece descrivono i campi elettromagnetici come ad esempio la luce. Quest'ultima era elemento di dibattito in quanto empiricamente con l'esperimento di Young era stato dimostrato che essa è sottoposta ai fenomeni di diffrazione e interferenza: tipici delle onde; ma l'effetto fotoelettrico (emissione di elettroni da parte di una superficie a seguito di un'illuminazione) era spiegabile solo se trattata come insieme di particelle.
La meccanica quantistica assume quindi che l'unico modo per spiegare ogni fenomeno fisico è il non limitarsi a considerare la luce \textit{o} onda \textit{o} particella, bensì accettare che essa è entrambi contemporaneamente.
Non è comunque solo la luce ad essere soggetta al dualismo, bensì ogni particella: ad esempio nell'esperimento di Zeilinger viene fatto notare come anche i neutroni (normalmente considerati particelle) presentano i fenomeni di interferenza
\subsection{Indeterminazione e casualità}
\subsection{La misura}
\section{Conseguenze apparentemente controintuitive}
\subsection{Indeterminazione di Heisenberg}
\subsection{Entanglement}
\subsubsection{Il gatto di Schrodinger: vivo o morto?}