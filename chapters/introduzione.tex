Sono ormai quasi cento anni dall'avvento della fisica moderna e dall'enunciazione delle leggi della meccanica quantistica. Similarmente quasi un secolo è passato dai primi computer e dalla nascita (con Alan Turing) dell'informatica.\\
Non è certamente un caso; infatti l'elettronica e il mondo digitale non esisterebbero senza meccanica quantistica in quanto l'elemento fondamentale, il transistor, è un componente che sfrutta, appunto le leggi quantistiche. In modo molto semplificato si può vedere come un pulsante che può lasciar passare corrente oppure no, che però non funziona meccanicamente aprendo e chiudendo fisicamente un circuito. Esso è infatti azionato da un segnale elettrico che modifica le proprietà quantistiche dei semimetalli con cui è costruito.\\
Senza transistor non si potrebbero gestire i bit [Capitolo 3] nelle memorie flash e soprattutto non si potrebbero creare le porte logiche [Capitolo 4].
Questi computer sono però chiamati comunque classici pur dipendendo dalla meccanica quantistica in quanto l'informazione gestita, cioè i bit, è classica.\\
Si distinguono quindi da essi i computer quantistici che non sfruttano la meccanica quantistica solo come una base necessaria per il funzionamento, bensì fanno in modo che l'informazione manipolata (i qubit) sia essa stessa soggetta alle leggi quantistiche. La ricerca in tale ambito è molto più giovane, ha circa 20 anni, ma ha le potenzialità di apportare miglioramenti e cambiamenti alla società al pari di quello che ha fatto e che sta facendo l'informatica classica.\\
Per questo (e per il mio interesse verso l'informatica e la meccanica quantistica) ho deciso di trattare come argomento personale i Computer e la Computazione quantistica concentrandomi soprattutto sull'opposizione con le macchine di Turing, cioè il modello alla base dei computer classici.